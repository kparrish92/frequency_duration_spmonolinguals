% Options for packages loaded elsewhere
\PassOptionsToPackage{unicode}{hyperref}
\PassOptionsToPackage{hyphens}{url}
%
\documentclass[
  12pt,
]{article}
\usepackage{lmodern}
\usepackage{amsmath}
\usepackage{ifxetex,ifluatex}
\ifnum 0\ifxetex 1\fi\ifluatex 1\fi=0 % if pdftex
  \usepackage[T1]{fontenc}
  \usepackage[utf8]{inputenc}
  \usepackage{textcomp} % provide euro and other symbols
  \usepackage{amssymb}
\else % if luatex or xetex
  \usepackage{unicode-math}
  \defaultfontfeatures{Scale=MatchLowercase}
  \defaultfontfeatures[\rmfamily]{Ligatures=TeX,Scale=1}
\fi
% Use upquote if available, for straight quotes in verbatim environments
\IfFileExists{upquote.sty}{\usepackage{upquote}}{}
\IfFileExists{microtype.sty}{% use microtype if available
  \usepackage[]{microtype}
  \UseMicrotypeSet[protrusion]{basicmath} % disable protrusion for tt fonts
}{}
\makeatletter
\@ifundefined{KOMAClassName}{% if non-KOMA class
  \IfFileExists{parskip.sty}{%
    \usepackage{parskip}
  }{% else
    \setlength{\parindent}{0pt}
    \setlength{\parskip}{6pt plus 2pt minus 1pt}}
}{% if KOMA class
  \KOMAoptions{parskip=half}}
\makeatother
\usepackage{xcolor}
\IfFileExists{xurl.sty}{\usepackage{xurl}}{} % add URL line breaks if available
\IfFileExists{bookmark.sty}{\usepackage{bookmark}}{\usepackage{hyperref}}
\hypersetup{
  pdftitle={The effect of lexical frequency on word duration: analyzing corpus data in Spanish and English},
  hidelinks,
  pdfcreator={LaTeX via pandoc}}
\urlstyle{same} % disable monospaced font for URLs
\usepackage[margin=1in]{geometry}
\usepackage{graphicx}
\makeatletter
\def\maxwidth{\ifdim\Gin@nat@width>\linewidth\linewidth\else\Gin@nat@width\fi}
\def\maxheight{\ifdim\Gin@nat@height>\textheight\textheight\else\Gin@nat@height\fi}
\makeatother
% Scale images if necessary, so that they will not overflow the page
% margins by default, and it is still possible to overwrite the defaults
% using explicit options in \includegraphics[width, height, ...]{}
\setkeys{Gin}{width=\maxwidth,height=\maxheight,keepaspectratio}
% Set default figure placement to htbp
\makeatletter
\def\fps@figure{htbp}
\makeatother
\setlength{\emergencystretch}{3em} % prevent overfull lines
\providecommand{\tightlist}{%
  \setlength{\itemsep}{0pt}\setlength{\parskip}{0pt}}
\setcounter{secnumdepth}{-\maxdimen} % remove section numbering
\usepackage{tipa}
\usepackage{xcolor}
\usepackage{booktabs}
\usepackage{longtable}
\usepackage{array}
\usepackage{multirow}
\usepackage{wrapfig}
\usepackage{float}
\usepackage{colortbl}
\usepackage{pdflscape}
\usepackage{tabu}
\usepackage{threeparttable}
\usepackage{threeparttablex}
\usepackage[normalem]{ulem}
\usepackage{makecell}
\usepackage{xcolor}
\ifluatex
  \usepackage{selnolig}  % disable illegal ligatures
\fi
\newlength{\cslhangindent}
\setlength{\cslhangindent}{1.5em}
\newlength{\csllabelwidth}
\setlength{\csllabelwidth}{3em}
\newenvironment{CSLReferences}[2] % #1 hanging-ident, #2 entry spacing
 {% don't indent paragraphs
  \setlength{\parindent}{0pt}
  % turn on hanging indent if param 1 is 1
  \ifodd #1 \everypar{\setlength{\hangindent}{\cslhangindent}}\ignorespaces\fi
  % set entry spacing
  \ifnum #2 > 0
  \setlength{\parskip}{#2\baselineskip}
  \fi
 }%
 {}
\usepackage{calc}
\newcommand{\CSLBlock}[1]{#1\hfill\break}
\newcommand{\CSLLeftMargin}[1]{\parbox[t]{\csllabelwidth}{#1}}
\newcommand{\CSLRightInline}[1]{\parbox[t]{\linewidth - \csllabelwidth}{#1}\break}
\newcommand{\CSLIndent}[1]{\hspace{\cslhangindent}#1}

\title{The effect of lexical frequency on word duration: analyzing
corpus data in Spanish and English}
\author{}
\date{\vspace{-2.5em}}

\begin{document}
\maketitle

The present study investigates the effect of lexical frequency on word
duration in Spanish. Previous studies have found that vowel duration in
English is influenced by extra-linguistic factors, such as lexical
frequency (e.g., Gahl, S. 2008; Gahl 2009; Lohmann 2018). The shortening
of frequent forms relative to infrequent ones may correspond to
articulatory routinization, which suggests that neuromotor routines
become more reduced with repetition (Bybee 2001; Newmeyer 2006).
However, evidence showing that in homophone pairs (e.g., \emph{thyme} --
\emph{time}), the item with higher frequency is shorter than the
infrequent one reveals that frequency effects on duration may not be due
to repetition of a phonological form but to lemma frequency effects
instead. In the case of Spanish, differences in vowel duration are not
as prominent as in English. Although vowel duration in Spanish
unstressed syllables tends to be shorter than in stressed syllables
(Gálvez 1994), it is still unclear whether lemma frequency modulates
duration in Spanish.

The present study aims to, first, replicate the frequency effect found
in vowel duration in English (Gahl, S. 2008; Gahl 2009; Lohmann 2018)
but in whole duration. Second, the present study explores the effect of
lexical frequency on word duration in Spanish. This study analyzes
English corpus data from the \emph{Free ST American English Corpus} and
Spanish corpus data from the \emph{Open SLR Corpus}. The English data
consisted of 2806 recordings of cellphone conversations, and the Spanish
data consisted of 1928 recordings of XXX conversations?. Lexical
frequency was measured using the XXX. The data was analyzed using two
Bayesian linear regressions with duration as the outcome variable and
speech rate and orthographic length as fixed predictors.

The results replicated the frequency effects previously found in
English. English frequent words were found to be shorter than infrequent
ones when orthographic length and speech rate were controlled for
(Figure 1, left panel). In Spanish, results exhibited a negative linear
relationship between lexical frequency and word duration (Figure 1,
right panel). Frequent words were shorter than infrequent ones when
orthographic length and speech rate were controlled for. The findings
have implications for neo-generative (Levelt 1989), exemplar (Johnson
2006), and hybrid (Pierrehumbert 2016) models of sound representation.

\clearpage

\textbackslash begin\{figure\}
\includegraphics[width=1\linewidth]{/Users/juanjogp/Desktop/frequency_duration_spmonolinguals/docs/abstracts/NewSounds22/figs/joined_plots}
\textbackslash caption\{Whole word duration in English (Left Panel) and
Spanish (Right Panel) as a function of lexical frequency and
orthographic length (length\_z) with the most plausible line of best
fit.\}\label{fig:plot-joined} \textbackslash end\{figure\}

\begin{center}
References
\end{center}

\begingroup
\setlength{\parindent}{-0.5in}
\setlength{\leftskip}{0.5in}

\phantom{.}

\textcolor{white}{\\} \vspace{-0.5in}

\hypertarget{refs}{}
\begin{CSLReferences}{1}{0}
\leavevmode\hypertarget{ref-bybee_2001}{}%
Bybee, Joan. 2001. \emph{Phonology and {Language} {Use}}. Cambridge
{Studies} in {Linguistics}. Cambridge: Cambridge University Press.
\url{https://doi.org/10.1017/CBO9780511612886}.

\leavevmode\hypertarget{ref-gahl_2008}{}%
Gahl, S. 2008. {``\emph{Time} and \emph{Thyme} {Are} Not {Homophones}:
{The} {Effect} of {Lemma} {Frequency} on {Word} {Durations} in
{Spontaneous} {Speech}.''} \emph{Language} 84 (3): 474--96.
\url{https://doi.org/10.1353/lan.0.0035}.

\leavevmode\hypertarget{ref-gahl_2009}{}%
Gahl, S. 2009. {``Homophone {Duration} in {Spontaneous} {Speech}: {A}
{Mixed}-Effects {Model}.''} \emph{Undefined}.
\url{https://www.semanticscholar.org/paper/Homophone-Duration-in-Spontaneous-Speech}.

\leavevmode\hypertarget{ref-galvez_1994}{}%
Gálvez, Rafael Marín. 1994. {``La Duración Vocálica En Español.''}
\emph{ELUA}, no. 10 (December): 213--26.
\url{https://doi.org/10.14198/ELUA1994-1995.10.11}.

\leavevmode\hypertarget{ref-johnson_2006}{}%
Johnson, K. 2006. {``Speech Perception Without Speaker Normalization.''}
In \emph{Talker Variability in Speech Processing}, edited by K Johnson
and J. W. Mullenix, 145--66. San Diego: Academic Press.

\leavevmode\hypertarget{ref-levelt_1989}{}%
Levelt, Willem J. M. 1989. \emph{Speaking: {From} {Intention} to
{Articulation}}. Edited by Maurice V. Wilkes. {ACL}-{MIT} {Series} in
{Natural} {Language} {Processing}. Cambridge, MA, USA: A Bradford Book.

\leavevmode\hypertarget{ref-lohmann_2018}{}%
Lohmann, Arne. 2018. {``Cut (n) and Cut (v) Are Not Homophones: {Lemma}
Frequency Affects the Duration of Noun--Verb Conversion Pairs.''}
\emph{Journal of Linguistics} 54 (4): 753--77.
\url{https://doi.org/10.1017/S0022226717000378}.

\leavevmode\hypertarget{ref-newmeyer_2006}{}%
Newmeyer, Frederick J. 2006. {``On {Gahl} and {Garnsey} on {Grammar} and
{Usage}.''} \emph{Language} 82 (2): 399--404.
\url{https://doi.org/10.1353/lan.2006.0100}.

\leavevmode\hypertarget{ref-pierrehumbert_2016}{}%
Pierrehumbert, Janet B. 2016. {``Phonological {Representation}: {Beyond}
{Abstract} {Versus} {Episodic}.''} \emph{Annual Review of Linguistics} 2
(1): 33--52.
\url{https://doi.org/10.1146/annurev-linguistics-030514-125050}.

\end{CSLReferences}

\endgroup

\end{document}
